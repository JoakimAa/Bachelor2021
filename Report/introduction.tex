 \chapter{Introduction}
\label{ch:intro}
A lot of companies have employees that travel in their work.
This travel is usually funded by their employer.
In order to keep track of all the transactions generated by this travel, the company will store the receipts from the
transactions.
By saving the receipts in this manner, it greatly reduces the effort needed from employers to refund their employees
for their purchases.

Going through these transactions manually is time-consuming and labor intensive.
Because of this, creating an automated solution for extracting the data contained in these receipts, will effectively
reduce the amount of manpower and manhours required by the company.
The goal of this thesis is to create an automated solution for this data extraction in the form of a machine learning model, receiving images and delivering extraced data via a RESI API\@.
\section{Background and motivation}\label{sec:background-and-motivation}
\subsection{Project Group}\label{subsec:project-group}
\textbf{Joakim Spjutøy Aarskog} is a 24-year-old computer science student at Østfold University College.
He has an interest in technology and photography.\\
\\
\textbf{Sondre Lillelien} is a 26-year-old computer engineering student at Østfold University College.
Interests include technology, engineering and AI.\\
\\
\textbf{Sander Riis} is a 23 years old computer science student at Østfold University College.
He likes to develop programs and learn about new technology.

\subsection{Employer}\label{subsec:employer}
The employer for this thesis is Simployer AS, a tech company that specializes in Human Resource Management (HRM).
They deliver HRM software to 15.000 customers and have over 1.2 million users in Norway and Sweden.
Simployer is passionate about giving other employers and their employees the tools to be able to effectively organize themselves.
Simployer is a sizable company with over 250 employees, and a history going back 35 years.
Our main contact in Simployer is Flemming Ottosen, their CTO\@.

\subsection{Problem statement}\label{subsec:problem-statement}
As Simployer delivers HRM software to many Scandinavian companies, they have a large database containing receipts
generated by their employees.
These documents are stored in an unorganized manner, making manual validation of these documents difficult and time-consuming.
The documents are typically in the form of an image of a receipt taken with a mobile camera.
The documents lack metadata, which makes searching through them to find a specific document difficult.
Simployer would like a way to organize this data automatically and attach relevant metadata to the documents.
This metadata includes the price of the transaction, the company the service was purchased from, and the date of the transaction.
In addition to attaching metadata to existing documents, Simployer would like to automatically generate and attach metadata to new documents coming in to their database.

\subsection{Motivation}\label{subsec:motivation}
Artificial intelligence and machine learning is a rapidly growing field in computer science.
The biggest obstacles that AI and ML faced in the past, was a lack of computing power and a lack of data.
In today's world, both of these obstacles have largely been overcome as computing power keeps growing and data is a
more abundant resource than before.
The motivation for this thesis is to take advantage of this abundance of computing power and data, to solve real life
problems.

\subsection{Objectives}\label{subsec:objectives}
The objectives for this thesis are:
\begin{description}
    \item[Objective 1] To create a machine learning model, that can identify desired data in an image of a receipt.
    \item[Objective 2] To have the machine learning model be capable of improving it's performance when new data is acquired.
    \item[Objective 3] To develop an API that can deliver data to the machine learning model and fetch the results.
\end{description}

\subsection{Method}\label{subsec:method}
The aim of the thesis has pivoted somewhat since the start of the semester.
Originally the goal was to create a machine learning model that can correctly identify desired data from a large variety of image types with different quality.
The goal is still to create such a model, but more focus will be directed towards the learning module.
Therefore, we have de-prioritized being able to work with a large variety of images.

The methods used for this thesis are the following:
\begin{compactitem}
    \item Deep learning literature research
    \item Text analysis and text extraction articles
    \item Clearly defining the scope and objectives of the thesis
    \item .NET Core articles
    \item Workflow management in Jira
\end{compactitem}

\subsection{Deliverables}\label{subsec:deliverables}
The following items should be delivered by the thesis' end-date:
\begin{compactitem}
    \item A report detailing the entire thesis
    \item A machine learning model that can extract desired data from the images it is given
    \item An API that can deliver images to the ML pipeline and request the results of data extraction done on the images
\end{compactitem}

\section{Report outline}\label{sec:report-outline}
Chapter 2 explains the thesis' topic and scope in detail.
In addition;
the methods used to complete the work, along with literature the work is based on will be included here.

Chapter 3 provides a detailed walkthrough of how the thesis is designed.
The design is made to efficiently and correctly tackle the problem statement.

Chapter 4 describes the implementation of the thesis.
This includes how the thesis group chose to implement the design discussed in previous chapters, and the research methods used to arrive at design decisions.

Chapter 5 presents the results the thesis group's work in an objective manner.

Chapter 6 discusses the results discovered in chapter 5.
This includes how the results differ from what was expected, how the findings might be relevant to the thesis' field of study, etc.

Chapter 7 concludes and summarizes the thesis in a manner that can be read by someone who did not read the entire report.





