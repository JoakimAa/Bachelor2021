\chapter{Summary and conclusion}
\label{ch:summaryandconclusion}
This chapter provides a brief summary of the entire report, so that someone who has not read the entirety of the report will still be able to catch the main points.
In addition, the chapter will include a conclusion on the state of the software created for this project, and whether the desired goals were met.

\section{Summary}\label{summary}
\subsection{Overview}\label{overview}
The goal of this project was to create machine learning software capable of extracting key data from images of receipts.
This key data is in the form of the receipts date, total price and type in the form of company name.
The images are delivered to the machine learning models via a .NET REST API, through a web client.
The web client receives the extracted data from the machine learning models via the same API.
A secondary goal of the project was to store the images uploaded via the API in a database, so it can be used to further improve the machine learning models.

\subsection{Design}\label{design}
The figures \ref{fig:pipeline} and \ref{fig:ML API} provide a visual overview of the software pipeline created for the project.
The design of the software pipeline is split into two main parts, the API and the data extraction modules.
Once the user has requested data extraction by uploading an image via the web client, the API stores the image in a database, so it can potentially be used for further training of the models.
The data extraction modules are made up of three machine learning models.
These three models include a CNN trained by the project group, an OCR used for image-to-text conversion, and NER software implemented using Spacy.
The CNN is used as an image classifier to determine the company that issues the receipt.
This CNN is trained using a dataset of receipts provided by the project employer.
The OCR software is used to convert the receipt image into text, in order to extract the date and to allow the NER software to work on a text input.
The NER software is used to determine the total price in the receipt, by checking words labeled as money by Spacy.
A python API implemented with the Flask framework is used to pass data between the models, and to communicate with the .NET API.

\subsection{Implementation}\label{implementation}
The .NET REST API is implemented with the MVP principle and is written in C\#.
The CNN model is created using Tensorflow and Keras, and is trained using a dataset of 1500 images.
The dataset originally provided to us by Simployer had a large variety of different receipts, with very few receipts of the same type.
In order to allow us to train our CNN to recognize a smaller amount of receipts, we had to cut down on the dataset variety.
This was done by selecting the three most common occurrences of receipts founds when going through the dataset, and using image augmentation software to generate new images of those receipts.
Before using the image augmentation software, the newly created subset of the original dataset had only about 100 images.
After using the image augmentation software, the size of the dataset increased to 1500 images.

OCR GREIER

SPACY GREIER

\subsection{Results and discussion}\label{resultsanddiscussion}


Beskrivelse av kapttel

Kort oppsummering

Mål og ble mål nådd

Forbedringer
