\chapter{Design}
\label{ch:design}

\begin{figure}[h]
    \center{\includegraphics[width=1\textwidth]{Images/pipeline}}
    \caption{Graphic illustrating the project pipeline (PLACEHOLDER)}
    \label{fig:figure3}
\end{figure}
The project is designed to be able to take an input image of a receipt sent over http, and provide metadata to that image based on it's contents.
This metadata includes the company name that issued the receipt, the date the receipt was made, and the price of the purchase.
In order to transfer the image to our machine learning models, a .NET API is used.

API GREIER HER

We have decided to split up the metadata extraction into two separate parts;
a Convolutional Neural Network for image classification and a Recurrent Neural Network for natural language processing.

The CNN will take an image as input, decide which company issued that receipt based on previous training data.

The RNN will take text as input.
In order to extract the text from the images, an open-source OCR is used.

Finally, the outputs of the CNN and RNN and combined and provided back to the user.
This pipeline is illustrated in figure 3.1

\section{API}\label{sec:API}

\section{Pre-processing}\label{sec:pre-processing}
A CNN does not require images of large resolution in order to accurately classify an image.
In addition, training the network with large resolution images take significantly longer.
Because of this, all the images that are going to be fed into the CNN are downscaled.
This also solves the problem of images being of different sizes, as the CNN requires all the images to have the same size.

\begin{figure}[h]
    \center{\includegraphics[width=1\textwidth]{Images/beforeafterrescale}}
    \caption{Before and after image rescaling}
    \label{fig:figure3.2}
\end{figure}
While the text in the image is now completely unreadable, the CNN will have no issues telling images in this format apart from each other.

\section{Convolutional Neural network}\label{sec:CNN}
As stated previously, we are using a CNN as an image classifier in order to determine which company issued the receipt.

\begin{figure}[h]
    \center{\includegraphics[width=1\textwidth]{Images/cnnplaceholder}}
    \caption{Graphic illustrating the Convolutional Neural Network (PLACEHOLDER)}
    \label{fig:figure3.3}
\end{figure}

Figure 3.3 illustrates the layout of our CNN.
The input layer takes the pixel value of the image.
Because of this, the number of nodes in the input layer has to be equal to the amount of pixels in the image.
The output layer has three nodes, one for each type of receipt the network is trained to classify.

We are using a supervised learning-model to train our network.
Because of this, the training data has to be labeled.
This labeling can be labor intensive if the dataset you are labeling is large.
Since we started with a very small amount of images before the use of data augmentation, it is a small task.

\section{Optical Character Recognition}\label{sec:OCR}



\section{Recurrent Neural Network}\label{sec:RNN}

