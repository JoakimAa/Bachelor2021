\cleardoublepage
\chapter{Analysis}
\label{ch:analysis}

\section{Research topic}\label{sec:research-topic}

Our project is to create a solution that allows a user to take an image of a receipt and automatically upload the relevant data to a database.
This automates the process of handling travel receipts, thus making it less time-consuming and reduces the cost.
To achieve this, we will be using Optical Character Recognition software and machine learning\@.
The extracted data from the image consists of date, total amount, and the receipt type.

The app will consist of two main parts: The OCR, and the model that reads the data.
We are going to use a free open source OCR, named Tesseracts.
The OCR is what will allow us to extract text from the images.
This OCR will send all the text that is on the receipt in the form of an array to the model.
The model will then decide what information is the correct data to extract and keep.

With the app we also need to include an API that will talk with the app and talk with Simployers server.
This API gets the data from the model and will prompt the information that is selected for the user for a final
validation, before sending it to the Simployer database.
This will allow Simployer to easily search in the database based on the metadata that was extracted.

\section{Tools}\label{sec:tools}
This section details the different tool that have been considered for use in the thesis.
\subsection{OCR engines}\label{subsec:ocr-engines}
There are many OCR engines that could be used for this thesis.
Three main ones have been selected for consideration.
These are Google Cloud Vision API, Amazon Textract, and Tesseract.
\subsubsection{Google Cloud Vision API}\label{subsubsec:API_Google}

Google Cloud Vision API contains a well optimised OCR and good variety of detection.
The detection features are listed in figure 2.1.

\begin{figure}[h]
    \center{\includegraphics[width=1\textwidth]{Images/googleAPI_features}}
    \caption{Overview of features (list from https://cloud.google.com/vision/pricing)}
    \label{fig:figure1}

\end{figure}

We can see that Google API offers everything from text recognition to landmark location and crop Hints.
The Google API is know to be industry leading in the field, so this comes as no surprise.
This API will have almost everything you need to develop an application with text recognition.

The API also has a lot of features in their beta version which is not included in figure 2.1.
\clearpage


\begin{figure}[h]
    \center{\includegraphics[width=1\textwidth]{Images/googleAPI_prices}}
    \caption{Overview of prices (list from https://cloud.google.com/vision/pricing)}
    \label{fig:figure2}

\end{figure}

Google is the best in the field, but they are also the most expensive API to use.
The prices are calculated for every 1000th request
This means that you are getting the first 1000-requests free and then you will need to play 1.5\$ for each 1000-request you do on for example text detection.
If you have a request size of 50.000-requests this will be 49*1.5\$ = 73,5 dollars (620 NOK).
This will reset every month.

\subsubsection{Amazon Textract}\label{subsubsec:API_Amazon}

Amazon's AWS hosts a service called Textract.
This cloud-based service lets you upload an image to their text recognition software.

\textbf{Detect Document Text API}
This feature is the classic OCR feature that will extract text from a document.
Amazon will charge you 0.0015\$ per page (1.50\$ per 1000) for the first million pages, when you exceed 1 million the price wil go down to
0.0006\$ per page (0.6\$ per 1000)

\textbf{Table Extraction}
This is an improved version of Detect Document Text API. This will group the text in tables.
This is helpful with structured data, such as financial reports.
Amazon will charge you 0.015\$ per page (15.0\$ per 1000) for the first million pages, when you exceed 1 million the price wil go down to
0.01\$ per page (10\$ per 1000)

\textbf{Form Extraction}
This detects key-value pairs in documents, for example if your document has a field named "first name" it will pair it will Mike.
This way "first name" would be the key and "Mike" would be the value.
This makes it easy to either sort it into a database or reuse the values in variables.
With normal OCR the relation is lost.
Amazon will charge you 0.05\$ per page (50\$ per 1000) for the first million pages, when you exceed 1 million the price wil go down to
0.04\$ per page (40\$ per 1000)

\textbf{Analyze Document API}

This is a combination of the tables and forms extraction
Amazon will charge you 0.065\$ per page (65\$ per 1000) for the first million pages, when you exceed 1 million the price wil go down to
0.05\$ per page (0.50\$ per 1000)

Note that Amazon changes their pricing on the region you are in (you might want to check the price for your region), they also do not include a free 1000 scans per a month.

\subsubsection{Tesseract}\label{subsubsec:Tesseract}
Tesseract is an open-source OCR engine that supports over 100 different languages.
It has the reputation of being one of the best open-source OCR engines.
It was created by HP in the 80's and later made open-source in 2005 and funded by google since 2006.
Most other OCR APIs today are build on top of this, like Google Cloud Vision API.

If this engine is used, code that takes advantage of the engine's built-in methods must be written.
When dealing with high-quality images and computer generated pdf receipts, writing code to extract the text is relatively simple.
But when the images are of lower quality, more care must be given to pre-processing to improve the accuracy of the text extraction.
This is because OCR software needs good contrast and little to no nice in the images.
Another problem is curved text, like in a book.
This might be the hardest ting to solve.

\section{Preprocessing}\label{sec:preprocessing}
Having images of ideal quality for use in an OCR is a luxury.
Often you will have images of low quality or taken from a suboptimal angles.
In order to mitigate the effects that low quality images will have on the text extraction accuracy, the images should go through a preprocessing stage before getting sent to the OCR software.

\subsection{Scaling the image}\label{subsec:scaling-the-image}
OCR software will have the best performance when given images between 300dpi and 600dpi.
Anything less will make it unreadable for the OCR and anything more will extra processing power with little to no improvement in accuracy.

\subsection{Skew correction}\label{subsec:skew-correction}
OCR engines use line segmentation to separate the text found in an image into different lines of data.
Therefore, it is important that the OCR is given images that are as straight as possible.
Skew correction is the part of the preprocessing pipeline that addresses this issue.

\subsection{Remove background noise}\label{subsec:remove-background-noise}
Image noise can reduce the accuracy of the OCR engine.
Because of this, removing noise in the background is very important to improve readability.
Applying a Gaussian blur is one way of reducing the noise in the image.

\subsection{Binarization}\label{subsec:how-to-create-contrast}
OCR engines are usually designed to handle input images with a black-and-white color scheme.
The process of converting a colorized or grayscale image to black-and-white is called binarization.
Once an image has gone through binarization, the background should be white and any foreground elements like text should be black.

If your image is in RGB format, it will first need to be converted to grayscale format before thresholding techniques can be used.
Adaptive thresholding is one such technique.
It involves partitioning the image into smaller regions, and calculating the threshold for what should be black and what should be white, in that region.



