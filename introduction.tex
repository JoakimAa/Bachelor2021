\chapter{Introduction}
\label{ch:intro}
A lot of companies have employees that travel in their work.
This travel is usually funded by their employer.
In order to keep track of all the transactions generated by this travel, the company will store the receipts from the transactions in order to refund their employees for their purchases.

Going through these transactions manually is time-consuming and labor intensive.
Because of this, creating an automated solution for extracting the data contained in these receipts will effectively reduce the amount of manpower and manhours required by the company.
This aim of this project is to explore and discuss these automated solutions.

\section{Background and motivation}\label{sec:background-and-motivation}
\subsection{Project Group}\label{subsec:project-group}
\textbf{Joakim Aarskog} is a 24-year-old computer science student at Østfold University College.
He has an interest in technology and photography.\\
\\
\textbf{Sondre Lillelien} is a 26-year-old computer engineering student at Østfold University College.
Interests include technology, engineering and AI.\\
\\
\textbf{Sander Riis} is a 23 years old computer science student at Østfold University College.
He likes to develop programs and learn about new technology.

\subsection{Employer}\label{subsec:employer}
The employer for this project is Simployer AS, a tech company that specializes in Human Resource Management (HRM).
They deliver HRM software to 15.000 customers and have over 1.2 million users in Norway and Sweden.
Simployer is passionate about giving other employers and their employees the tools to be able to effectively organize themselves.
Simployer is a sizable company with over 250 employees, and a history going back 35 years.
Our main contact in Simployer is Flemming Ottosen, their CTO.

\subsection{Problem statement}\label{subsec:problem-statement}
As Simployer delivers HRM software to many Scandinavian companies, they have a large database containing receipts generated by their employees.
These documents are stored in an unorganized manner, making manual validation of these documents difficult and time consuming.
The documents are typically in the form of an image of a receipt taken with a mobile camera.
The documents lack metadata, which makes searching through them to find a specific document difficult.
Simployer would like a way to organize this data automatically and attach relevant metadata to the documents.
This metadata includes the price of the transaction, the company the service was purchased from, and the date of the transaction.
In addition to attaching metadata to existing documents, Simployer would like to automatically generate and attach metadata to new documents coming in to their database.

\subsection{Motivation}\label{subsec:motivation}
Artificial intelligence and machine learning is a rapidly growing field in computer science.
The biggest obstacles that AI and ML faced in the past was a lack of computing power and a lack of data.
In today's world, both of these obstacles have largely been overcome, as computing power keeps growing and data is a more abundant resource than ever.
The motivation for this thesis is to take advantage of this abundance of computing power and data to solve real life problems.

\subsection{Objectives}\label{subsec:objectives}
In order to solve the problem presented by the employer, several solutions will be discussed and evaluated.
We will not aim to create a finished working solution that is ready for use by Simployer.
The objective for this thesis is:
\begin{description}
    \item[Objective 1] To explore possible solutions that handle the problem of assigning metadata to documents.
    \begin{description}
        \item[Objective 2.1] To attempt to create a machine learning model that is capable of differentiating between different types of documents.
        \item[Objective 2.1] To train the model using existing image data provided by Simployer
        \item[Objective 2.1] To explore methods to improve the model using new image data uploaded by users.
    \end{description}

\end{description}

\subsection{Method}\label{subsec:method}
The aim of this thesis has changed over the course of the semester.
Initially, the goal was to deliver working software in the form of an API.
This API was supposed to solve the issues described in the problem statement.
In conversation with Simployer, the aim has shifted to create a simple prototype, as well as discuss and explore the possible ways to solve the problem.
In research methods used for this thesis are the following:
\begin{compactitem}
    \item Deep learning literature
    \item Text analysis and text extraction articles
    \item .NET Core articles
    \item
\end{compactitem}

\subsection{Deliverables}\label{subsec:deliverables}
The following items should be delivered by the projects end-date:
\begin{compactitem}
    \item A thesis report detailing the entire project
    \item A prototype API created in .NET Core framework
    \item A machine learning model that can differentiate between different types of receipt documents.
\end{compactitem}

\section{Report outline}\label{sec:report-outline}



