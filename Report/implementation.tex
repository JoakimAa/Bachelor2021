\cleardoublepage
\chapter{Implementation}
\label{ch:implementation}
Chapter 4 describes the implementation of the thesis.
This includes how the thesis group chose to implement the design discussed in previous chapters, and the research methods used to arrive at design decisions.

\section{REST API}\label{sec:REST API}

The API exposes several endpoints that allows for saving an image and a receipt to a database.
The API is also connected to the ML and OCR module.

\section{Web solution}\label{sec:Web solution}

The web solution is made with React in JavaScript.
The purpose of the web solution is to give the user an interface where they can upload an image of a receipt.
And get the corresponding data from the image, and then send in the receipt with the data from the image.
This is to make the process of entering the data less time consuming.

When the user uploads an image, the web solution sends a POST request to the API, then the API saves the image to
the database.
When the image is saved, the image is sent to the OCR for text recognition.
If the output makes sense the data is returned and displayed to the user.
Then if the returned data is correct the user can then upload the receipt to the database.

\section{Convolutional Neural Network}{sec: cnn}

Our CNN is implemented using Keras.
Keras is a library for tensorflow that allows for the creation of neural network models with only a few lines of code.
The model consist of an input layer, two hidden layers, and an output layer.
The input layer and first hidden layer are convolutional layers with 64 nodes in each layer.
The second hidden layer and the output layer are dense layers, where the hidden layer has 64 nodes and the output layer has 3 nodes.
The model uses the Adam optimizer and the categorical crossentropy loss function.







